% This example An LaTeX document showing how to use the l3proj class to
% write your report. Use pdflatex and bibtex to process the file, creating 
% a PDF file as output (there is no need to use dvips when using pdflatex).
% Modified 

% This dissertation was built upon base template provided.

\documentclass{l3proj}

\begin{document}

\title{Team I: ResDiary Restaurant Recommendation System}

\author{Vladimir Bardarski \\
        Paulius Dilkas \\
        Domantas Jurkus \\
        Eduard Kalfov \\
        Josh O'brien \\
		Joseph O'Hagan}

\date{31st March 2017}

\maketitle

% ##################################################
% LAST EDIT: 	06/03/17	Joseph
% ##################################################
\begin{abstract}
The abstract shall go here! Here is some things to keep in mind while writing it.
\end{abstract}

\begin{itemize}
\item The abstract is likely the first substantive description of your work read by an examiner. View it as an opportunity to set accurate expectations.
\item The abstract is a summary of the whole thesis. It presents all the major elements of your work in a highly condensed form. (Write it having written the rest of the paper) The paper sets the abstract.
\item It must be capable of substituting for the whole paper when there is insufficient time and space for the full text.
\item Keep it short and snappy. 
\item The primary function of your thesis (and by extension your abstract) is not to tell readers what you did, it is to tell them what you discovered.
\item Approximately the last half of the abstract should be dedicated to summarizing and interpreting your results.
\item The most common error in abstracts is failure to present results.
\end{itemize}

% Comment out this line if you do not wish to give consent for your work to be distributed in electronic format.
% We hereby give consent - spread the knowledge - pending on result of project
\educationalconsent
\newpage

%==============================================================================

% ##################################################
% LAST EDIT: 	06/03/17	Josh
% ##################################################
\section{Introduction}
\label{sec:intro}
% An introduction, explaining the purpose of the document, a very brief outline of the project and a summary of the structure of the rest of the document (approximately 1-2 pages).

The Professional Software Development (PSD3) course at The University of Glasgow requires students to engage with the practices and methodologies used in modern large-scale software engineering. The purpose of this dissertation is to document the development of the software project created as part of this course by Team I. 

% ---------- A ----------
The project was to build, over the course of several months, a recommendation engine for the Glasgow-based company ResDiary.
The main deliverable was a system capable of producing sensible restaurant recommendations for existing ResDiary users; a system that could be integrated into the existing ResDiary platform at a later date. 

Our team consisted of six third-year Computing Science students. Within the group there was a broad range of skills, interests and experience - with two members actively working as software professionals, and another having participated in an internship. For some members, however, this was a first opportunity to interact with a real client. 

In this document we outline, in detail, the entire process: from the initial requirements gathering with our customer, through to final system delivery. 

In section \ref{sec:background} we present the background to the project, the motivations of the customer and how we arrived at the agreed deliverables.

%this will surely be expanded to enumerate each separate section better I would like to discuss in detail the practices, issue-tracking, the team work/team load - Josh etc.

In subsequent sections \ref{sec:alice} through Section \ref{sec:reflections} we explore the challenges we faced through development and the steps we took to resolve them, explore the impact of team dynamics on the outcome and reflect on what we have learned from the experience. We also explain how we applied the good development practices learned in PSD3. In particular we highlight the role of version control, agile development and issue tracking.

\newpage

% ################# Comment Log ####################
% 	A. 	We mention the integration but the final 
%		state of the system is for it to be used
%		as a proof of concept / prototype by RD.		<--	Joseph 
% ##################################################

%==============================================================================

\section{Case Study Background}
\label{sec:background}
% A description of the case study background and context. This should include a description of the project customer (what was the nature of the organisation you were working for), their objectives for the project, and a summary of what was actually achieved. Where appropriate, this section should also make reference to similar related projects in order to make the context clear (approximately 4-5 pages).

% ##################################################
% LAST EDIT: 	18/03/17	Josh
% ##################################################
\subsection{Customer}
\label{sec:customer}
% The customer organisation and background.

% Here we want answer the question of who are ResDiary and what do they do.
% Additionally answer who played the customer role of ResDiary to us on the project.

% Who are ResDiary?
ResDiary are a Glasgow-based online restaurant reservation service; a commercial organisation providing a comprehensive, easy to use booking and table management platform for use by the hospitality industry. The company provides 24-hour reservation services through both social media and their own booking portal ResDiary.com. Diners can browse restaurants, book tables and place reviews. Restaurants can access tools which let them optimize their yields, manage their reservations and attract new diners. Their global service sees 9.7 million bookings every month, and their platform is used by over 6,500 restaurants across 58 countries. 

ResDiary senior software engineers Adam Connelly and Ian Strachan acted as customer representatives throughout the duration of the development. They helped us understand the company line-of-thought behind the project, in addition to providing useful feedback, answering queries and supplying the team with the anonymised ResDiary booking data we required.

% ################# Comment Log ####################
% Mention their names? you sure? - Dom
%	-> Other dissertations did it so sure - Joseph
% Served as SLAVES - Dom
% ##################################################

% ##################################################
% LAST EDIT: 	18/03/17    JOSH
% NOTE: 		Reference papers on Amazon / Netflix Models / Netflix Prize
% NOTE:         Added more specifics on Netflix Recommendations - make sure to tie in to our final output
%               Need to demonstrate that the research was RELEVANT
% ##################################################
\subsection{Customer Objectives And Rationale}
\label{sec:custobjectives}
% The rationale and initial objectives for the project.

% Initial Meeting and the customer's motivation for the project.

The initial customer meeting occurred on October 19th and was led by Ian Strachan. This was our first contact with the customer and served as the customer requirements elicitation meeting. The meeting began with an overview of the ResDairy business \ref{sec:customer}, a discussion of the services they provide and an explanation of their technology stack. The ResDiary daily operation involves gathering large quantities of valuable customer data. This collection of big data, however, is currently unused beyond supporting basic business needs (i.e. retrieving booking records). The developers view this as a significant shortcoming of their system.

As such, therefore, the company is exploring potential ways to exploit this vast quantity of valuable data. The developers had conducted some research into the area, and discovered that none of their competitors currently offer a restaurant recommendation service within their booking platform. Thus, if they were able to develop a system which would make recommendations to users based on their previous dining habits and similarity to other users, they would gain a competitive edge. It would also help increase restaurant discovery on their platform, which is beneficial to both restaurants and potential clientele. 

The inspiration for the idea stems from the similar system provided by online services such as Amazon and Netflix, which push recommended products and films respectively to their users. The Netflix model, in particular, was the closest reference point for the system they wished to develop. A parallel between making recommendations based on a user's film history and their similarity to other users, and a similar recommendation engine using ResDiary's dining history is easy to see. As a starting point for our own research, they suggested looking into the Netflix Prize - a competition held by Netflix starting in 2006 challenging participants to better their own recommendation algorithm. 

The goal of the algorithm was to accurately predict a user's rating of a film based solely on previous user ratings and no additional information about either the users or films. A set of training data (a subset of Netflix's real user data) with half a million user ratings was provided to participants, with the algorithm using this data to predict user ratings for a disjoint subset of the user data. No information other than user ID, film ID and date of rating was provided. Our Resdiary project followed a similar structure. ResDiary's users, as with Netflix's, leave review scores ranging from 1 to 5. We were given a subset of their user data which we split into further subsets for training and for evaluation. The research provided an excellent springboard from which to launch our own collaborative filtering model. With such a strong similarity between the requirements of our system and theirs, it was a natural choice.

We also spent time discussing what the customer viewed as the end state of the system - whether it would be integreated into the existing ResDiary portal or whether the output would suffice as a proof-of-concept prototype. They were initially undecided in this regard, partially due to an internal transition in their own development frameworks, and thus suggested our initial aim should be to focus on the creation of the recommendation engine. The decision regarding the final state of the project ultimately would not be made until midway through the development cycle when the customer decided to view the project as simply a proof-of-concept. Nonetheless, at all points during development, due consideration was given to the future integration of our system with ResDiary. 

% ################# Comment Log ####################
% ##################################################

\subsection{Project Scoping}
\label{sec:ourinitobjectives}
% ##################################################
% LAST EDIT:  18/03/17  Josh
% NOTE:         JOSH: Expanded on importance of requirements.
% NOTE:         References: importance of requirements gathering, cost of making corrections 
% ################################################## 
The risks and costs associated with a project of this nature demand thorough background investigation and cogent planning before development proper commences. The cost of correcting errors grows enormously in the latter stages of a software project.

Our first major task was to formalise our discussions with the client and produce a requirements specification to serve as a project proposal document. The goal was to have a clear outline of the scope of the project and a set of deliverables to present to the client at the next meeting on November 16th. Our requirements gathering occurred in tandem with conducting the necessary initial background research on machine learning and recommendation systems. 

The proposal document included an array of requirements that the team agreed upon, based on our interpretation of the customer's initial project pitch. It was important that our requirements were reasonably comprehensive and realistic, but we accepted that they would be continually revised and refined as the project developed. 

\subsubsection{Initial functional requirements:}
\begin{itemize}
\item The recommendation engine must accurately suggest restaurants based on the users' dining history and similarity to other users with similar eating preferences.
\item Recommended restaurants should be in close proximity to where the user typically eats or the geographical location of where they are currently searching.
\item The recommendation engine may recommend restaurants that  the user has previously visited should the user allow this option.
\end{itemize}

\subsubsection{Initial nonfunctional requirements:}
\begin{itemize}
\item The engine should be written to allow for easy integration into the existing ResDiary system.
\item The system should give a response within one second after receiving the request (provided data is stored locally).
\item New users should be presented an optional quick questionnaire to gather initial data.
\item User and restaurant locations should be interpreted using coordinates rather than city name as those are of arbitrary precision within the dataset.
\end{itemize}

To help us understand the actual scenarios in which the system might be used, we prepared a set of user stories (ranked by priority). They provided a quick, intuitive way to ensure we had covered all the conceivable use cases the customer may require. We also prepared a high-level system UML diagram and a step-by-step work-flow of how the system would generate the actual recommendations. Due to the customer's ambiguity regarding the final state of the system, the endpoint was left intentionally vague to allow for flexibility. Instead the emphasis was to build and produce the most accurate recommendation for a given user. 

% ################# Comment Log ####################
% If required (page count) maybe cut down requirements engineering and incorporating customer feedback in reflection point - we did it pretty well though there isn't much to improve in that regard
% ##################################################

\subsection{Customer Refinements To Initial Objectives}
\label{sec:custrefineinitobj}
% ##################################################
% LAST EDIT:  15/03/17  Joseph
% NOTE:         Needs reworking
% NOTE:			Needs referencing
% NOTE:			Needs cutting down page count 
% ##################################################

% The suggested alternative was to use a nightly build system which the team would utilise in the final version of the system (!R! REFERENCE TO SPARK REFLECTION !R!).

Presenting our proposal to the customer, they felt we had a good grasp of their vision as they agreed with the proposed functional requirements and could envision how our high level system would operate and integrate into their existing one. In particular they expressed interest in the potential to “fine tune” the system through altering the significance place on individual recommenders. Concern was expressed though with the (!!! proposed response rate / local computation of recommendations !!!) as the customer felt the need to further clarify the volume of data the system would be expected to work with in a real world deployment. The suggested alternative was to use a nightly build system as this was their expected from the system would operate under in a real world setting. In addition they suggested developing a lightweight front end application to display the recommendations. They rationalised its throwaway nature with the with the belief that such an application would help them to better understand the system, assist with demonstrating the functionality of the system and provide a clear indication of if the system could produce sensible results. 

Incorporating this this feedback into the specification  the team felt the need to revise the nonfunctional requirements of the project. This saw the removal of the aforementioned “1 second local response rate” (!R! REFERENCE ORIGINAL NONFUNCTIONAL REQUIREMENTS !R!) in place of the following two requirements:

\begin{itemize}
\item Provided the data is not stored locally, the system should be setup to allow for nightly updates to the recommendations made.
\item Have the ability to “fine tune” the recommendation engine by altering the weighting significance of different components of the recommendation such as distance, price, 
reviews, etc.
\end{itemize}

Furthermore a soft goal developed within the team to produce a front end application to showcase the recommendation system. While some of the team felt this justified being defined within the specification the majority instead felt the focus of the project should be on creating the most accurate recommendations and delay defining an end state until it was defined by the customer. Should the customer not provide clarity earlier it was decided to press for clarity on this issue during the scheduled January 26th meeting and define a agreed handover state for the project.

Prior to the January meeting the development efforts were split between the creation of this throwaway application and on the long term solution to the problem. The prototype system was also used during an interim meeting on December 7th where the technical decisions of the long term solution discussed at length as they were being finalised and implementation of them commenced (!R! REFERENCE TECHNICAL REFLECTIONS !R!).  

% ################# Comment Log ####################
% If required (page count) maybe cut down requirements engineering and incorporating customer feedback in reflection point - we did it pretty well though there isn't much to improve in that regard
% ##################################################

% \subsection{Technical Research \& Prototyping}
% \label{sec:techresearchproto}
% ##################################################
% LAST EDIT:  15/03/17  Joseph
% NOTE:         Perhaps best to merge with above subsection
% ##################################################

% ################# Comment Log ####################
% ##################################################

\subsection{Defined End State of System}
\label{sec:jandefinedstate}
% ##################################################
% LAST EDIT:  15/03/17  Joseph
% NOTE:         Needs reworking
% NOTE:			Needs referencing
% NOTE:			Needs cutting down page count 
% ##################################################

Upon presenting the progress made at the January 26th meeting the team pressed the customer regarding what their vision of the final state of the system. Through this discussion it became clear the customer wished to view the project as “proof-of-concept” with their intended use being to assess the worth of creating a similar system for their existing system. Additionally they wished to learn from the system, as they have little experience in this particular field, and should they believe a similar system to be of use to their business they wished to use our system as a prototype to justify the development resources required. 

With the final handover state of the system known, the team decided a redesign of the front end of the application was necessary. (!R! TIE IN / REFERENCE BENEFITS OF STARTING AFRESH !R!). The primary justification for the redesign coming through an observation made while giving the demonstration at the January 26th meeting to a non-technical member of the ResDiary team. Through this non-technical perspective it was clear that the system was not properly communicating the recommendations being made as it was unclear to a non-technical user how the system generally operated, if it was functioning correctly and if the results it produced were sensible. As the end state of the system was now clearly defined end state the following functional requirements were discovered and added to the project with the aim of producing a front end redesign which visually showed the system was operating correctly and producing accurate predictions:

Functional
\begin{itemize}
\item The front end display should display recommendations for a random pool of users to simulate typical use of the system.
\end{itemize}

Nonfunctional
\begin{itemize}
\item The front end be designed such that it is aesthetically clear that the recommendations made are sensible and accurate.
\end{itemize}

With the addition of the above requirements the decision was made in the aftermath of the January 26th meeting to drop the proposed nonfunctional requirement of implementing recommendations for new users. A more detailed account into this decision can be found in Section (!R! REFERENCE DROPPED FEATURE REFLECTION POINT !R!) though the general belief was that such a feature was of particular interest to the customer and thus should be dropped from the intended implementation of the project.

% ################# Comment Log ####################
% ##################################################

% ##################################################
% LAST EDIT: 	06/03/17	Joseph
% ##################################################
\subsection{Delivered Software}
\label{sec:finsoftware}
% Information on the final software that was delivered to the customer.
This section can only be written after the software is in its final handover state.
\newpage

% ################# Comment Log ####################
% ##################################################

%==============================================================================
\section{Alice}
\label{sec:alice}

This is a example of how to include an image from the figures directory.

\input{figures/alice}

This is an example of how to reference an inlcuded figure (see Figure \ref{fig:alice}).

%==============================================================================
\section{Reflections}
\label{sec:reflections}
% ##################################################
% LAST EDIT: 	14/03/17	Joseph
% NOTE:	Currently misc. thoughts / notes
% ##################################################

% Several sections that reflect on your experiences during the team project. Each section should discuss one theme, characterised by incidents or events that occurred during the team course of the project from which you learned (approximately 12-15 pages).

As for the general structure of Section 3 the way I envision it. Regarding the order I envision starting by saying the team utilised Scrum and then reflect on whether Scrum was the right choice for the final reflection point / conclusion. The order of the other points can be changed around for best flow:
\begin{itemize}
\item SCRUM Overview - start with an overview stating the team used Scrum and maybe a brief sprint summary with reflection references or mini-reflections. The sprint summary may not be necessary as a good deal of context is provided above in relation to the development of the project.
\item Technical Decisions - this will be three sub reflections regarding 1. Choice of language 2. Choice of collab / content models 3. Choice of Spark
\item Testing - a section on testing and whether a better testing strategy could have been used.
\item Dropped Features - features were dropped and this raises the question why? So of the reasons were our own fault and some were not. This is worth hitting on and also provides a transision into reflection on our failures for why the features were cut from the project.
\item Team Structure - this was a significant shift in how the team internally thought of the project and is worth mentioning.
\item Communication and Task Backlog - communication came up again and again in retrospectives and so should feature.
\item Code Reviews - code reviews came up and were the thing our project really lacked as far as markers are concerned. Would also have prevented several issues the team faced during the development process.
\item Scrum vs XP / etc - was using the Scrum system the correct system? This I feel leads naturally into the conclusion from the third section as its a reflection that can tie in the other reflection points as well. Ties them all together and leads out of the reflection section somewhat naturally.
\end{itemize} 

Start by stating that the team decided to use the Scrum methodology and reference reflection point of if this was the correct system to use. Potentially give iteration overview (short summary of each iteration) with either a mini-reflection or reference to further reflection point for all key points which occurred.
\begin{itemize}
\item Sprint 1 - importance of getting the requirements correct - risks \& costs of fixing, etc (MINI REFLECTION).
\item Research conducted and technical decisions - reference to reflection points.
\item Dropped Feature - reference to reflection point.
\item Testing - reference to reflection point.
\item Front end redesign - throwaway code and prototyping (MINI REFLECTION)
\item Pair Programming - reference to reflection point 
\item ETC
\item ETC
\end{itemize}


\subsection{REFLECTION POINT - Requirements Engineering}
\label{sec:teamstructure}
% ##################################################
% LAST EDIT:  18/03/17  Joseph
% NOTE:         Temp notes / thoughts
% ##################################################

In this regard I believe the team did fairly well as we had a good grasp of the customer's expectations and desire for the system. We made some suggestions that they themselves had not though of (rerecommend locations / new user recommendations via quiz / restrictions on recommendations (i.e. only recommend vegan or restaurants)) though the majority of the suggested additional features did not see implementation due to poor time management for some and striving too far away from the customer's vision for others.

Potential spin here is to say that the team realised the importance of the requirements engineering stage and the importance of making the correct design decision in relation to the risks and costs assoicated with jumping into development without planning, selecting the wrong system design, etc.

\begin{itemize}
\item (Lecture 13) Requirements gathering is an on-going iterative process that runs concurrently alongside requirements analysis and capture - our project saw this in real effect due to the changing nature of the requirements of our project.
\item (Lecture 14) Requirements engineering is an iterative process of elicitation, capture and validation - talk about how we did this to ensure we were gathering the correct requirements.
\item (Lecture 3) Causes of software project failing - building system for wrong reason, building the wrong system, building the system wrong. We could tie in the dropped new user recommendation to a proposed goal which was later determined to be a requirement which would be built for the wrong reason. As such the feature was dropped to avoid building an incorrect system.
\end{itemize}

% ################# Comment Log ####################
% ##################################################


\subsection{FINAL REFLECTION POINT - Scrum vs XP - Was Scrum The Right Choice}
\label{sec:scrumvsxp}
% ##################################################
% LAST EDIT:  15/03/17  Joseph
% NOTE:         Temp notes / thoughts
% ##################################################
This would be a reflection on using the Scrum system in comparison to another system such as Extreme Programming which might have been more suited to our project given its vague nature at the start and would also have helped combat the issues of code reviews / testing.

We ended up with this weird cross system that was mainly Scrum.

Probably pick one then compare against the other and determine if that would have worked better. Probably closer to Scrum of the above two.

Misc points:
\begin{itemize}
\item Time between meetings served as a good sprint dates - working on smaller iterations may have ensured better communication within the team.
\item Development typically did not change much within a sprint - XP difference here may not have made much of a difference though the dropped feature could have been explored earlier in the development cycle than the 2nd semester.
\item Task priority was a bit of an issue as we didn’t really do proper task estimation - XP would have solved that issue as it forces strict policy and task estimation.
\item Scrum has no engineering practices whereas our project definitely should have. XP would have forced pair programming from the start (rather than sporadic occurrences within sprints), ensured better, more thorough testing from an earlier stage in the project and avoided the whole code review “you deleted all my code you bastard” arguments.
\end{itemize}

% ################# Comment Log ####################
% ##################################################


\subsection{REFLECTION POINT - Team Structure}
\label{sec:teamstructure}
% ##################################################
% LAST EDIT:  15/03/17  Joseph
% NOTE:         Temp notes / thoughts
% ##################################################

REFLECTION POINT - Anarchy into Structure (Team Structure)
Initially the team had no real structure and instead utilised a developer anarchy team structure whereby someone would work on whatever system of the felt like at a given time. This eventually stopped working as the team subdivided into divisions - front end, back end and testing with back end being further divided into collaborative filter, evaluator, content based, NEO recommender.

Anarchy would explain the lack of time estimates and product owner however.

Reasons for the failure of the anarchy include the following reasons:

\begin{itemize}
\item Failure to communicate between the team - people weren’t discussing what they were actively working on, tickets didn’t reflect what people were working on, people didn’t yield results in an efficient enough manner.
\item Failure to fully understand domain - anarchy structures require huge expertise of the problem domain which regardless of how smart you are you cannot gain a grasp of in a few weeks. There’s a reason we pay for years worth of experience in a field / industry.
\item Motivation took a hit over Christmas (it was the holidays)
\item Lack of trust between team as this was the first time we worked together as a team on a long term project.
\item (Lecture 6) The sports model of role based development structure - the team adopted this model in the second half of development.
\item (Lecture 6) Are any of the other models such as Laissez-faire suitable?
\item (Lecture 6 / Lecture 7) The team paid measure to the Mythical Man Month text in order to avoid too many cooks entering the back end development after the team split to focus on the front and back end split.
\end{itemize}

Changing to a more structured development system did have an effect on productivity (testing improved - we actually had some, front end redesign was rad, more defined roles resulted in better communication within the team)

% ################# Comment Log ####################
% ##################################################


\subsection{REFLECTION POINT - Code Reviews \& Branching \& Pair Programming}
\label{sec:codereviewbranch}
% ##################################################
% LAST EDIT:  15/03/17  Joseph
% NOTE:         Temp notes / thoughts
% ##################################################
We did not do code reviews and this resulted in some ‘tensions’ during the development of the project. Code reviews from the start of the development would have been beneficial for many reasons (see papers).

Misc Points:
\begin{itemize}
\item The system was designed such that branching was unnecessary for the project. Branching was somewhat discouraged throughout the project in order to reduce merge costs / time. Whether the time saved outweighs the time spent fixing and altering other people’s code remains to be seen.
\item Lack of code reviews meant that the code often had inconsistent styling.
\item Lack of code reviews meant a significant portion of time was spent fixing or altering other people's code to work with parts of the system they did not realise they had broken.
\item Lack of code reviews meant code which was WIP was deleted prior to it being fully implemented into the system - again lack of branching issue somewhat.
\item Suggested at the first retrospective but ended up avoiding. Would have reduced a number of problems with development cycle but you can't be perfect.
\item Highlighted in retrospective 3 with the code integration being highlighted.
\item Highlighted even more in retrospective 4 with code integration really being highlighted.
\item Reference potential methods or strategies for code reviews which could have integrated into the development.
\item Reference how other and often professional developers recommend doing code reviews.
\item Mention that pair programming was conducted and the benefits of doing it. Mention it should have been conducted more frequently, reference that it was beneficial when suggesting if XP was potentially better than Scrum.
\item Code reviews might have improved development efficency as less time spent refactoring and fixing broken code. Reduced arguments caused due to deleting other people's code prior to full implementation. Branching also would have fixed this as only get back to master upon passing all tests. This would require testing to have been more efficent as well thus suggesting test strategy should have been utilised to improve test efficency.
\item (Lecture 26) Outline of inspection methods - should probably mention which we should have used in the code reviews.
\item (Lecture 8) Change management and the conflicts caused from developers working in parallel and contributions are made to the master branch which are not fully implemented or break some aspect of the system. In hindsight this probably should have justified a separate branch which would be integrated into the master upon passing all of the tests with a successful build in Jenkins the continuous integration system.
\item Pair programming done for design / initial coding of front end of application. Similarly done for aspects of the back end development of the project - well somewhat pair programming where one guy would bounce ideas and assist another in the structure and functionality of that particular aspect of the project. Potential spin for the front end is that a more experienced front end developer Josh worked with a less expereienced developer Joseph and the benefits of this (see note on pair programming on wiki somewhere for similar argument)
\end{itemize}

% ################# Comment Log ####################
% ##################################################


\subsection{REFLECTION POINT - Technical Overview Section}
\label{sec:techoverviewreflection}
% ##################################################
% LAST EDIT:  18/03/17  Joseph
% NOTE:         Temp notes / thoughts
% ##################################################
I see this being an overview of the technical decision making process with the three major technical decisions (Dev Language, Models, Spark) being sub-categories and reflection points to themselves. This is mainly concerned with the process of making decisions as a whole (the Python decison was made poorly according to a retrospective whereas the others were not). Very brief but might flow into the technical section better.

Technical decisions are a major factor of the development cycle, especially for a project such as ours, here is generally how we did it and here we present the three major techical standpoints of our project and reflect on the specifics of those individually...

\begin{itemize}
\item (Lecture 13) Avoid early commitments to particular design solutions during requirements elicitation - there are dangerous assoicated with early commitment to the wrong design decision. We were at risk of that so spent the time at the outset ensuring we were developing the right system based off our research of the various models.
\end{itemize}

% ################# Comment Log ####################
% Perhaps this will go but I like the idea of a
% short transision reflection to bridge the gap
% technical reflection and process reflections.
% This is sort of a hybird reflection that is both
% technical and non-technical.
% ##################################################


\subsection{REFLECTION POINT - Python}
\label{sec:pyreflection}
% ##################################################
% LAST EDIT:  15/03/17  Joseph
% NOTE:         Temp notes / thoughts
% ##################################################
Misc Points:
\begin{itemize}
\item What alternatives did we consider - C Sharp / Scala
\item What were the benefits of using it - already familiar / libraries / industry standard
\item The choice of Python is mentioned in a retrospective as the choice was slow and not well discussed and we just jumped into it - how did we ensure this did not occur with other major technical decisions
\item Did Python influence other later choices - Spark, etc.
\item What were the limitations of using Python
\item Was using Python the correct choice in the long term 
\item Reference papers and libarires to justify why Python 
\end{itemize}

% ################# Comment Log ####################
% ##################################################


\subsection{REFLECTION POINT - Models}
\label{sec:modelreflection}
% ##################################################
% LAST EDIT:  15/03/17  Joseph
% NOTE:         Temp notes / thoughts
% ##################################################
Misc Points:
\begin{itemize}
\item Explain why collaborative filter
\item Explain why content based filter
\item Reference papers and other models (Netflix) to justify choice
\item What were the limits of the models - customer asked to only retrain on new data but limit of model disallows this - need for specific type of data that is not always present within the data set.
\item Did we choose the right models
\item Did the models impact later decisions
\item Probably suggest some alternative models which could have been explored
\end{itemize}

% ################# Comment Log ####################
% ##################################################


\subsection{REFLECTION POINT - Apache Spark}
\label{sec:sparkreflection}
% ##################################################
% LAST EDIT:  15/03/17  Joseph
% NOTE:         Temp notes / thoughts
% ##################################################
Misc Points:
\begin{itemize}
\item Team decided the entire team should spend time over Christmas playing around with Apache Spark to ensure common understanding of the system between team members. This may not have been the best use of resource and instead team could have split earlier and the front end developers not bothered playing around with it as they did not end up touching it directly during the development of the project. 
\item Common knowledge was gained though but a more typical front end, back end split in focus may have been beneficial and resulted in an improved front end application sooner.
\item Why was the decision made to use Spark?
\item Did any previous decisions (models / languages / tools) influence our choice of Spark
\item What were the alternatives to using Spark
\item What were the benefits of using it?
\item What were the limitations of using it?
\item Was Spark the right choice?
\item What impact did deciding to use Spark have on the project?
\end{itemize}

% ################# Comment Log ####################
% ##################################################


\subsection{REFLECTION POINT - Dropping Features}
\label{sec:droppingreflection}
% ##################################################
% LAST EDIT:  15/03/17  Joseph
% NOTE:         Temp notes / thoughts
% ##################################################

Misc note from section 2:
While this particular feature had been viewed as a bonus feature from the outset of development the team felt based on the customer’s feedback that the customers were not particularly interested in such a feature. Research conducted earlier had shown that making recommendations of this type were primarily a search based recommendation whereas the customer’s interests lay instead in recommendations based on similarity between users. As such the development resources were instead focused on the aforementioned goals of producing a more accurate collaborative and content based recommendation system and the production of a high quality front end to professionally display system output. 
\begin{itemize}
\item Recommendation reasons dropped due to lack of time.
\item Proof of better than random suggestions also dropped.
\item New user recommendations dropped due to feature creep as it was not the customers desire and we suggested it and added it to the project. Additionally it is not a recommendation made off similarity between users rather a search based filter. This was found during research phase and development halted there which was good.
\item (Lecture 16) Planning poker would have been good to have been played and task cost estimation should have been under constant review as new information is uncovered and work on the project develops - this would have been another use for frequent stand up meetings.
\end{itemize}
% ################# Comment Log ####################
% ##################################################


\subsection{REFLECTION POINT - Communication Breakdown \& Task Backlog}
\label{sec:communicationbreakdown}
% ##################################################
% LAST EDIT:  16/03/17  Joseph
% NOTE:         Temp notes / thoughts
% ##################################################
This was an issue which occurred throughout most of the retrospectives.

Misc Notes:
\begin{itemize}
\item Slack was set up and made the dedicated communication channel for project communication.
\item Despite having a Slack communication remained somewhat poor.
\item Decision made to ensure tickets actively represent what feature of the system you are working on.
\item Ticket system improved but commits / development did not occur in a time efficient manner.
\item Sub teams additionally helped communication within front end and back end through team communication as a whole was still somewhat lacking.
\item Specific incident - Edward not providing data sent from customer in a timely manner.
\item Part of team felt agreeing to meet up in person at least once a week to work on project was a waste of time / resources - this had an adverse effect on communication that wasn’t addressed. Instead majority of team met up once a week at scheduled time as some developers strayed through the valley of the shadow of death into the unknown.
\item Perhaps Joseph focusing less on development taking on a product owner style of role would have improved the development.
\item Talk about the benefits a product owner would have had on the development of our specific project - improved efficiency of tackling task backlog (tasks were avoided from one milestone to the next)
\item (Lecture 7) Perhaps burn down charts per sprints would have been a good idea and would have improved prediction of task completion, backlog management, communication and prioritisation of backlog tasks.
\end{itemize}


\subsection{REFLECTION POINT - Testing}
\label{testing}
% ##################################################
% LAST EDIT:  16/03/17  Dom
% NOTE:       TODO
% ##################################################
A suite of unit tests was written to ensure that each commit did not break the execution of the project.

% What testing model was used?

% Was it easy/hard to test and why?

The unit tests were set up to run on Jenkins, a continuous integration environment.
Jenkins was set up to push notifications to the team's communication channel whenever a test case would fail. This enabled early notification of issues, allowing the team to patch up bugs early on.
Jenkins also provided coverage reports for the source files, indicating the percentage of classes, conditionals and lines covered.
Overlap between testing and evaluation.
Although there were considerations on testing the front-end portion of the project, it was agreed that such tests were not essential for the presentation of the recommendations.

Integration testing:
The System recommender was tested in conjunction with other recommenders for integration.

Regressiong testing:
A regression test for hyperparameter learning was written.

What could have been improved:
A notable shortcoming was that tests were not written under any particular testing strategy. This lead to the discovery of several modules missing tests cases, and some tests using untested modules. Following a stricter test plan would have led to more systematic test development, allowing to uncover the beforementioned problems earlier.
Mutation tests were considered but not used.

% ################# Comment Log ####################
% ##################################################

\subsection{REFLECTION POINT - MISC POINTS FROM PSD NOTES}
\label{sec:miscpsd}
% ##################################################
% LAST EDIT:  16/03/17  Joseph
% NOTE:         Temp notes / thoughts
% ##################################################
This is just a collection of quotes and notes taken from the PSD notes that will likely be included as reflection points. These are some of the key PSD points which we'll need to hit on with the reflections PSD literature.
\begin{itemize}
\item (Lecture 10) Continuous integration practices minimise the disruption caused by rapid, concurrent changes to software systems - explain why this was suitable for our purposes, concurrent development of aspects of the system (evaluator and recommender), etc <-- Testing / Code Reviews
\item (Lecture 19) Throw-away front end prototype was used and was good as it gave a better understanding of the requirements the front end would require upon it becoming the final state of the system. This knowledge and the discovered failures fed into the redesigned system and so the throw-away prototype was very beneficial to the project. (MINI REFLECTION POINT ?)
\item (Lecture 19) Prototyping is used to reduce risks caused by uncertainty in software projects, not introduce more risk - to some extent this was done for components of the back end system as well.
\item (Lecture 30) Formal specification is a specific thing so probably don't just refer to a specification document as being formal loosely.
\item (Lecture 32) Refactoring is an important process which went on over the course of development.
\end{itemize}
% ################# Comment Log ####################
% ##################################################

%==============================================================================
\section{Conclusions}
\label{sec:conclusions}
% A conclusion that draws general and wider lessons from the case study (approximately 1-2 pages)

Explain the wider lessons that you learned about software engineering,
based on the specific issues discussed in previous sections.  Reflect
on the extent to which these lessons could be generalised to other
types of software project.  Relate the wider lessons to others
reported in case studies in the software engineering literature.

%==============================================================================
\bibliographystyle{plain}
\bibliography{dissertation}
\end{document}
